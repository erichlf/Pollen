Pollination is a critical component of every ecosystem, essential to creating
and maintaining diversity and reproduction, and required for world crop
production \cite{KleinEtAl2007}.  The two main vectors for pollen movement are
abiotic pollination where individual pollen grains are dispersed with through
the aid of wind and/or water, and biotic pollination where an animal (most
commonly an insect) carry grains from one individual plant to another.  Given
the size of individual pollen grains, direct monitoring of how pollen is
dispersed across the landscape is impractical.  Several indirect approaches have
been developed including the use of pollen traps and the application of genetic
paternity approaches applied to successfully pollinated seeds
\cite{BitzerPatterson1967,StreiffEtAl1999}.  While these approaches are able to
quantify the end result of the dispersal process, they provide no direct
information on the specifics of the transport mechanism.  This information is
critical to understanding gene flow for a variety of reasons including how to
better understand how different features of the landscape, which have variable
permeability, affect pollen movement \cite{DyerSork2001,DyerEtAl2012}. In this
study, we simulate both passive and active dispersion of pollen between plants,
and examine the consequences of movement assumptions and their interactions
with variation in plant density.

Generally pollen dispersal studies**(add citations here), for both abiotic and
biotic pollen dispersal, have assumed that pollen is distributed as a purely
random diffusion process.  While this may be a good assumption for abiotic
pollination, there is little evidence that animals randomly diffuse across the
landscape during pollination \cite{LevinKerster}.  In fact, there are several
examples of animal pollinators exhibiting \emph{trap line} behavior \cite[e.g.,
repeated sequential visits to individual plants]{OhashiThomson}.  Even for
pollinator species that do not trapline, their movement patterns do not resemble
pure diffusion \cite{Cresswell03}. One approach to describe animal movement is
through the use of correlated random walks (CRW).  In these models,
directionality is not random but is instead based on the distribution about the
direction taken in the previous step.  Models based upon CRW have been applied
to a wide range of animal movement processes across varying ecological contexts
\cite{Bartumeus07,Byers01}, using deterministic diffusion \cite{Klages}, and
fractional Brownian motion \cite{Enriquez} approaches.

In this paper, an agent-based model (ABM) is used to simulate both biotic and
abiotic pollination.  The model consist of moving agents, animals, which
pollinate static agents, plants.  While Movement is modeled using a correlated
random walk (CRW).  In particular for abiotic pollination we assume there is no
correlation between consecutive steps, where for biotic pollination we assume
there is correlation.

To better characterize the differences between biotic and abiotic pollination we
examine several statistics describing pollinator movement (\emph{average path
distance} and \emph{average maximum distance}) and those relevant to the plant
reproduction (\emph{average pollination distance}, \emph{average maximum
pollination distance}, and \emph{average weighted diversity of fathers}).  This
is done by varying the strength of the correlation between consecutive steps of
movement.

We show that on average pollination events occur at larger distances in
moderately and highly correlated random walks, as compared to purely random
walks.  This can results in higher genetic diversity among species with biotic
dispersal than with abiotic dispersal. ({\bf Is that observed, Rodney?}) This
suggests that care must be taken when studying plants that are pollinated by
animals and appropriate models must be utilized. 

We organize the paper in the following way: First in \cref{sec:Methods} we
discuss the assumptions and model parameters. Next, in \cref{sec:Results} we
discuss the model results, including statistical results. Finally, in
\cref{sec:Discussion} we discuss the implication of the model and its results
on the study of pollination.

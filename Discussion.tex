Pollination is a critical process to all life and is a very difficult one to study for a variety of reasons: pollen size, pollinator diversity, plant diversity, etc.  In this study we have used a single agent-based model with correlated random walk to simulate both abiotic and biotic pollination.
This simulation approach provides a powerful method for studying pollination given the logistical difficulties associated with direct field monitoring.
  
A striking result is increases in pollination distances, average and maximum, that occur with the biotic model versus the abiotic model.  In particular at both low and high densities we observe at least 29\% and as high as 96\% increases in the average distance pollen travels in moderately or highly correlated random walks over the purely random walks.  For the average maximum pollination distance the increases range from 44\% to 148\%.  This can lead to unexpected consequences especially if say an animal pollinated, genetically modified crop is placed a particular distance from an non-genetically modified crop based on a purely random diffusion model.  In this situation there could be unintended cross contamination.


The majority of models studying pollination have assumed a purely random diffusion process. There is
clear evidence that there are differences in statistics seen between plants pollinated through wind
dispersal from those pollinated through animal dispersal \cite{LevinKerster}.  We demonstrate here how large these differences can be.
{\bf Add some important biology language on the consequences of these observations.}

It is clear from the results that density plays a large role in the overall gene flow for a species.  However in most of the statistics we calculated how animals move is also very important.  Generally the statistics are affected by movement more at low densities and less so at high densities.
%As can be seen in the results section the magnitude of turning angle had varying degrees of effects
%over different plant densities and therefore pollination patterns predicted by a model assuming a
%purely random walk could be vastly different from a model assuming a correlated random walk.
%For
%high plant densities, the effects of correlated random walk was less pronounced than that of low
%plant densities, except for the \emph{average weighted diversity of fathers}. In the case of
%\emph{average weighted diversity of fathers} the affect of turning angle magnitudes were more
%pronounced for high densities. Therefore, although diffusion models for densely populated plant
%species may not vary greatly from models that assume a correlated random walk for \emph{average
%pollination distance} or \emph{average maximum pollination distance} they will vary significantly
%for \emph{average weighted diversity of fathers}. This has the affect of under estimating the
%diversity of pollination for high plant densities and animal dispersal as compared to similar plant
%densities and wind dispersal.

%The variation between correlated random walk and that of a purely random walk is significant at low
%plant densities for the statistics such as \emph{average maximum distance}, \emph{average
%pollination distance}, and \emph{average maximum pollination distance} and so for the case of low
%plant densities the assumption of a purely random walk may lend to bias in the analysis of
%pollination. Most studies to date have been conducted on small herbaceous plant species whose
%densities tend to be high. Even though most of the animals statistics presented were not greatly
%influenced by turning angle for high plant densities the average weighted diversity of fathers had was
%still greatly affected by the turning angle at these densities, and therefore an assumption of a
%purely random walk would be an inappropriate assumption and at any of the densities examined in this
%study. Therefore a correlated random walk may be a better approximation to animal movement.




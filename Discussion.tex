\textcolor{red}{I'm not quite sure how math discussion sections work typically.
In Biology we tend to blovate a lot here putting things into context and
figuring out why this is a valuable thing to be added to the literature.
However, I think that may not be the apporach you are trying to do.  As a
consequence, I just minimally edited this section after reworking the first
paragraph a bit. }

Each factor that influences the behavior of pollinators may thus have large
effects on the spatial distribution of pollination. Because plants rely upon
pollinators for reproduction, the way in which these animals move across the
landscape may have significant impacts on plant population structure and
diversity. Reductions in the distance of pollination evenets or the size of the
potential population contributing pollen can have significant impacts including
inbreeding, loss of genetic diversity, and local extinction.  As such,
understanding which parameters are most influential in models that accurately
describe pollination becomes a critical tool for both conservation and
management policies.  Adopting a simualtion approach, as presented here,
provides a powerful method for studying pollination given the logistical
difficulties associated with direct field monitoring.  By far, the most striking
result of these model is the extent to which devaitions from random diffusion
impact pollination distances and the consequences for paternity in the plant
populations. 

The majority of models studying pollination have assumed a purely random
diffusion process. There is clear evidence that there are differences in
statistics seen between plants pollinated through wind dispersal from those
pollinated through animal dispersal \cite{LevinKerster}.  It is demonstrated
here through an agent based correlated random walk that if an animal is not
moving in a purely random fashion, then important pollination statistics can be
dramatically affected. 

As can be seen in the results section the magnitude of turning angle had varying
degrees of effects over different plant densities and therefore pollination
patterns predicted by a model assuming a purely random walk could be vastly
different from a model assuming a correlated random walk. For high plant
densities, the effects of correlated random walk was less pronounced than that
of low plant densities, except for the \emph{average weighted diversity of
fathers}. In the case of \emph{average weighted diversity of fathers} the affect
of turning angle magnitudes were more pronounced for high densities. Therefore,
although diffusion models for densely populated plant species may not vary
greatly from models that assume a correlated random walk for \emph{average
pollination distance} or \emph{average maximum pollination distance} they will
vary significantly for \emph{average weighted diversity of fathers}. This has
the affect of under estimating the diversity of pollination for high plant
densities and animal dispersal as compared to similar plant densities and wind
dispersal.

The variation between correlated random walk and that of a purely random walk is
significant at low plant densities for the statistics such as \emph{average
maximum distance}, \emph{average pollination distance}, and \emph{average
maximum pollination distance} and so for the case of low plant densities the
assumption of a purely random walk may lend to bias in the analysis of
pollination. Most studies to date have been conducted on small herbaceous plant
species whose densities tend to be high. Even though most of the animals
statistics presented were not greatly influenced by turning angle for high plant
densities the average weighted diversity of fathers was still greatly affected
by the turning angle at these densities, and therefore an assumption of a purely
random walk would be an inappropriate assumption and at any of the densities
examined in this study. Therefore a correlated random walk may be a better
approximation to animal movement.


In this study, an agent-based model simulates the pollination of plants in a
forest using two interacting agents; one mobile agent, \emph{pollinators or
animals} and one static, \emph{plants}.  To determine how genes flow we keep
track of the order of the plants each pollinator interacts with to determine the
likely genetics of the seed creation. In the following subsections we discuss
the specific rules used in the model.

\subsection{Movement}

Movement in the model is done by the pollinators, which carry pollen from one
plant to another.  Each step of the pollinator's movement is conducted in two
stages: \emph{searching} and \emph{movement}.  First, the pollinator checks a
neighborhood of radius $r$ to see if there are any plants within the
neighborhood.  If there are one or more plants, the pollinator chooses the
closest.  If there are two or more which are equidistant from the pollinator, one
is randomly chosen.

If there are no plants within a distance $r$ from the current location of the
pollinator, the pollinator moves according to a correlated random walk.  For the
correlated random walk, the direction is chosen based on a probability
distribution centered about its current direction, see
\cref{fig:TurningAngle}.  In the simulations we vary the maximum turning
angle (AMT) {\bf Erich-Why didn't we call it MTA instead)} \textcolor{red}{This
was the standard notation used in previous papers.}.  We relate the
strength of the correlated random walk with the size of the turning angle.  So
for a purely random walk AMT is $180^{\circ}$ and there would be no correlation,
and as AMT decreases the correlation grows stronger.
{\bf Erich-purely random diffusion should be true for the uniform distribution
but not the normal distribution, what are the results in the graphs using?}
\textcolor{red}{I believe the only distribution used on AMT was the uniform
distribution. In the thesis both were used.}

The pollinator then takes a step with length between 0 and 1 distributed
uniformly in the new chosen direction.  This length is denoted by $s_j^{(i)}$,
which is the $j^{th}-$step taken by the $i^{th}$ pollinator.

\begin{figure}[h!]
  \begin{center}
  \begin{minipage}[b]{0.48\textwidth}
    \centering
    \includegraphics[scale=0.4]{Figures/UniformTADistribution.pdf}
    \subcaption{Uniform distribution} \label{sfig:Uniform}
  \end{minipage}
%  \begin{minipage}[b]{0.48\textwidth}
%    \includegraphics[scale=0.4]{Figures/NormalTADistribution.pdf}
%    \subcaption{Normal distribution} \label{sfig:Normal}
%  \end{minipage}
  \begin{minipage}[b]{0.48\textwidth}
    \centering
    \includegraphics[scale=0.5]{Figures/TurningAngle.pdf}
    \subcaption{Example path} \label{sfig:TurningAngle}
  \end{minipage}
  \end{center}
  \caption{Turning Angle for uniform distribution (\cref{sfig:Uniform}),
  %normal distribution (\cref{sfig:Normal}),
  and depiction of what a path may look like
  (\cref{sfig:TurningAngle}).}\label{fig:TurningAngle}
\end{figure}
{\bf Erich-Maybe in this figure we should just use the uniform distribution?}


Alternatively, if the pollinator is already at a plant, the pollinator picks a
random direction uniformly and takes a step with length $r+1$ distributed
uniformly.  This will ensure that the pollinator will not immediately return to
the same plant on the next step.

\subsection{Pollination}

When a pollinator is on a plant, it collects pollen, distributes pollen, and
consumes food. Each plant has a particular number of flowers, $\phi$, from which
a pollinator may obtain pollen. When a pollinator visits a plant it picks up
pollen from one or more flowers. The number of flowers from which a pollinator
can obtain pollen is determined by the total number of flowers on a plant, the
fraction of flowers in bloom at any one time ($a$), the number of times ($j$)
the plant has previously been visited by a pollinator, and the maximum fraction
of flowers available for pollination ($\eta$). The formula for the number of
total flowers available for visitation during a $k^{th}$ visit to the $j^{th}$
plant ($f_{j,k}$) is given by

\begin{equation}\label{flowers}
f_{j,k} = \phi \cdot a \cdot \eta^k.
\end{equation}

The amount of food eaten and the amount of pollen collected is proportional to
the number of visited flowers. Pollinators collect and eat pollen from each
flower they visit.  The amount of pollen collected and the amount of food eaten
is proportional to \cref{flowers}. Let $f^{\left(i\right)}_{j,k}$
be the number of flowers visited by the $i^{th}$ pollinator during the $k^{th}$
visit to the $j^{th}$ plant, then the amount of pollen consumed by the $i^{th}$
pollinator after $m$ plant visits is given by 
\begin{equation}
  c^{\left(i\right)}_m = \sum_{j=0}^{m} \beta f^{\left(i\right)}_{j,k},
  \label{limit}
\end{equation}
where $\beta$ is the proportionality constant for
the amount of pollen collected at a plant.  Each pollinator has a maximum amount
of food they will ingest, $c_{max}$, where if they eat that much food they will
stop searching.

The fraction, $\alpha$, of all flowers pollinated, and the associated
probability that a flower is pollinated, $\rho$, are related by the equation,
\begin{equation} \label{Prob}
  \alpha = \rho \cdot \hat{f}_k.
\end{equation}
Using \cref{limit,Prob} we can determine the probability
that a flower is pollinated, $\rho$, by the formula
\begin{equation*}
  \rho = \frac{\alpha}{\phi} \cdot \frac{1 - \eta}{a \cdot \eta}.
\end{equation*}
To determine where the pollen originated when a flower is pollinated, we
consider each flower previously visited but exclude those flowers from the same
plant as the flower being pollinated.    Self-pollination, is not considered,
since the likelihood of self-pollination is low due to mechanisms that impedes
self-pollination\textcolor{red}{citation is needed}. Each other flower
considered has an equal likelihood of pollinating the current flower, and a
flower is chosen at random.

\subsection{Time and Stopping Criteria}

The speed a pollinator travels ($v$) is constant, as well as the time spent on a
plant ($t_{plant}$).  The travel time for a pollinator is then given by the
formula
\[
  t^{\left(i\right)} = \frac{s^{\left(i\right)}}{v} + T^{\left(i\right)} \cdot t_{plant},
\]
where $T^{\left(i\right)}$ is the number of plants visited by the $i^{th}$
pollinator. If we let the maximum allowable travel time be $t_{max}$, then once
$t^{\left(i\right)} \geq t_{max}$ or $c^{\left(i\right)}_m \geq c_{max}$ the
pollinator is removed from the simulation. $t_{max}$ is based on the optimal
searching time during the day.  When all pollinators leave the simulation, the
simulation is terminated.

\subsection{Model Statistics}

To best explore the inherent differences between biotic and abiotic pollination
this study focuses on the effects of the strength of the correlated random walk
as well as the effects of plant density.

\begin{table}[h]
  \centering
\setlength{\extrarowheight}{15pt}
\begin{tabular}{|l|l|}
  \hline
  % after \\: \hline or \cline{col1-col2} \cline{col3-col4} ...
  Measure & Equation \\ \hline   \hline
  Average Path Distance & $\bar{s} = \dfrac{1}{b} \nsum\limits_{i=1}^b\,
    \nsum\limits_{j=1}^n s^{\left(i\right)}_j$ \\ \hline
  \multirow{2}{*}{\parbox{0.2\textwidth}{Average Maximum \\ Distance}} &
    $\bar{M} = \dfrac{1}{b} \nnsum_{i=1}^b \max\limits_j
        \sqrt{\left(x^{\left(i\right)}_{1,0} - x^{\left(i\right)}_{1,j}\right)^2
          + \left(x^{\left(i\right)}_{2,0} - x^{\left(i\right)}_{2,j}\right)^2}$
    \\ & \\ \hline
  \multirow{2}{*}{\parbox{0.2\textwidth}{Average Pollination \\ Distance}} &
    $\bar{p} = \dfrac{1}{n} \nnsum_{i=1}^{n}
        \left(\frac{1}{\tau^{\left(i\right)}} \nsum_{j=1}^{\tau^{\left(i\right)}}
            \sqrt{\left(x^{\left(i\right)}_1 - x^{\left(j\right)}_1\right)^2
            + \left(x^{\left(i\right)}_2 - x^{\left(j\right)}_2\right)^2}
        \right)$ \\ & \\ \hline
  \multirow{2}{*}{\parbox{0.2\textwidth}{Average Maximum \\
      Pollination Distance}} &
    $\bar{P} = \dfrac{1}{n} \nnsum_{i=1}^{n} \max\limits_j
      \sqrt{\left(x^{\left(i\right)}_1 - x^{\left(j\right)}_1\right)^2
        + \left(x^{\left(i\right)}_2 - x^{\left(j\right)}_2\right)^2}$ \\ & \\ \hline
  \multirow{2}{*}{\parbox{0.2\textwidth}{Average Weighted \\ Diversity of
      Father}} &
    $E = \dfrac{1}{n} \nnsum_{i=1}^n
      \dfrac{1}{\frac{1}{\left(\tau^{\left(i\right)}\right)^2}
      \sum_{j=1}^{\Delta\tau^{\left(i\right)}} F^2_{j,i}}$ \\ & \\ \hline
\end{tabular}
\caption{Equations}
\label{tab:eqn}
\end{table}

We calculate \emph{Average Path Distance} and \emph{Average Maximum Distance},
which are based upon the pollinators' movement and gives a sense how this
movement differs with changes in density and the maximum turning angle.  We also
calculate \emph{Average Pollination Distance}, \emph{Average Maximum Pollination
Distance}, and \emph{Average Weighted Diversity of Fathers}, which are based on
the pollination events occurring in the model.  These give a more direct results
in how gene flow is affected in the model.  The calculations of these statistics
are given in the \cref{tab:eqn}.

In these equations it is assumed that $b$ is the number of pollinators, $n$ is
the total number of plants, $(x_{1,0}^{(i)},x_{2,0}^{(i)})$ is the starting
location of the $i^{th}$ pollinator, $(x_{1,j}^{(i)},x_{2,j}^{(i)})$ is the
location of the $i^{th}$ pollinator after $j$ steps, $\tau^{(i)}$ is the total
number of seeds for the $i^{th}$ plant, $\Delta\tau^{(i)}$ is the number of
different fathers contributing pollen to the $i^{th}$ plant, and $F_{j,i}$ is
the number of times the $j^{th}$ father contributed pollen to the $i^{th}$
plant.
